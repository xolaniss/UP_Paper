\title{Regulation and bank lending in South Africa: A narrative index approach\thanks{The authors thank the South African Reserve Bank for their financial assistance. This paper is provided as part of the Bank's call for papers on the impact of prudential regulation on financial services. We also thank Sanelisiwe Hlatshwayo and Mzuvukile Skatshwa for invaluable research assistance.}}

\author {Xolani Sibande\footnote{Economic Research Department, South African Reserve Bank; Email: xolani.sibande@resbank.co.za.} \and
Dumakude Nxumalo\footnote{Department of Economics, University of Pretoria, Pretoria, South Africa; Email: dumakude.nxumalo@up.ac.za} \and
Keaoleboga Mncube\footnote{Department of Economics, University of Pretoria, Pretoria, South Africa; Email: keaolebogamncube@gmail.com} \and
Steve Koch\footnote{Department of Economics, University of Pretoria, Pretoria, South Africa; Email: steve.koch@up.ac.za} \and
Nicola Viegi\footnote{Department of Economics, University of Pretoria, Pretoria, South Africa; Email: viegin@gmail.com}}

\date{\today}
\maketitle

\begin{center}
\textbf{Abstract}
\end{center}

\begin{abstract}
The extension of affordable credit is key to financial inclusion, but it could lower financial sector stability. Macroprudential policy on the other hand, is designed to mitigate financial sector risk. Thus, it is possible inclusion and macroprudential regulations may oppose one another. This study estimates and contrasts the impact of these potentially contradictory regulations on the bank lending rates and volumes of South Africa's largest banks. Our results suggest that macroprudential policy is working as intended as it is associated with increases in interest rates on unsecured lending rates, decreases in short-term secured and mortgage lending rates.  Inclusion focused regulation is associated with increased bank lending rates in unsecured credit. We observe a decrease in the growth of unsecured lending for households and an increase in secured lending for corporates. As opposed to offsetting regulations, we find that the estimated impacts of financial inclusion initiatives largely overlap with those estimated for macroprudential policy.
\end{abstract}


\noindent\textbf{Keywords}: Bank lending, narrative methods, finance regulation\\
\textbf{JEL Codes}: G01, G18, G28, G32, G38
\newpage